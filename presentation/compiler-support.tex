\begin{frame}[fragile]{Instrumentation of memory accesses}
  \begin{minted}{ocaml}
let d1 () =
  a := 1;
  !b
  \end{minted}
  \par\noindent\rule{\textwidth}{0.4pt}
  \begin{columns}[T]
    \begin{column}{0.45\textwidth}
      %\mintcmm[highlightlines={3, 7}]{src/instrumentation_example0.cmm}
      \begin{minted}[highlightlines={3, 6},escapeinside=||]{lisp}
(function d1 (param)

|\xglobal\colorlet{FancyVerbHighlightColor}{red}| (store a 1)


 (load_mut b))
      \end{minted}
    \end{column}
    \begin{column}{0.55\textwidth}
      \begin{overprint}
        \onslide<2>
      \begin{minted}[highlightlines={2-3, 5-8}]{lisp}
(function d1 (param)
 (extcall "__tsan_write8" a)
 (store a 1)

 (extcall "__tsan_read8" b)
 (load_mut b))
      \end{minted}
      \end{overprint}
    \end{column}
  \end{columns}
\end{frame}


\begin{frame}[fragile]{Function entries and exits}
  Recall: TSan gives the backtrace of \textbf{both} conflicting accesses
  \bigskip
  \lstsetNone
  \lstset{frame=single,backgroundcolor=\color{black!6}}
  \begin{lstlisting}
==================
WARNING: ThreadSanitizer: data race (pid=3080294)
  Write of size 8 at 0x7f70feffebe0 by thread T1 (mutexes: write M90):
    #0 camlSimple_race.d2_274 simple_race.ml:7 (simple_race.exe+0x420a72)
    #1 camlStdlib__Domain.body_706 stdlib/domain.ml:211 (simple_race.exe+0x44119f)
    #2 caml_start_program <null> (simple_race.exe+0x47d1a7)
    #3 caml_callback_exn runtime/callback.c:197 (simple_race.exe+0x4461eb)
    #4 domain_thread_func runtime/domain.c:1167 (simple_race.exe+0x44a383)

  %*\begin{tikzpicture}[remember picture,overlay] \node (marker0) {}; \end{tikzpicture}*)Previous read%*\begin{tikzpicture}[remember picture,overlay] \node (marker1) {}; \end{tikzpicture}*) of size 8 at 0x7f70feffebe0 by main thread (mutexes: write M86):
    #0 camlSimple_race.main_277 simple_race.ml:13 (simple_race.exe+0x420b36)
    #1 camlSimple_race.entry simple_race.ml:34 (simple_race.exe+0x420fcf)
    #2 caml_program <null> (simple_race.exe+0x41ffb9)
    #3 caml_start_program <null> (simple_race.exe+0x47d1a7)
[...]\end{lstlisting}
  %
  \begin{tikzpicture}[remember picture,overlay]
    %\draw[rectangle,minimum height=10pt,fit=(marker0) (marker1),color=red];
    %\draw[color=red] (marker0) to[bend left=90] (marker1);
    %\draw[color=red] (marker1) to[bend left=90] (marker0);
    \node[draw,fit=(marker0) (marker1),color=red,yshift=2pt,rounded
      corners=5pt,ultra thick] {};
  \end{tikzpicture}
\end{frame}

\begin{frame}[fragile]{Function entries and exits}
  \begin{minted}{ocaml}
let d1 () =
  a := 1;
  !b
  \end{minted}
  \vspace{-15pt}%
  \par\noindent\rule{\textwidth}{0.4pt}
  \vspace{-15pt}%
  \begin{columns}[T]
    \begin{column}{0.45\textwidth}
      \begin{minted}[highlightlines={2, 7}]{lisp}
(function d1 (param)

 (extcall "__tsan_write8" a)
 (store a 1)

 (extcall "__tsan_read8" b)
 (load_mut b))
      \end{minted}
    \end{column}
    \begin{column}{0.55\textwidth}
      \begin{overprint}
      \onslide<2>
      \begin{minted}[highlightlines={2, 7-9}]{lisp}
(function d1 (param)
 (extcall "__tsan_func_entry" return_addr)
 (extcall "__tsan_write8" a)
 (store a 1)

 (extcall "__tsan_read8" b)
 (let res (load_mut b)
   (extcall "__tsan_func_exit")
   res))
      \end{minted}
      \end{overprint}
    \end{column}
  \end{columns}

  \small
  \begin{itemize}
    \item To be able to show \emphA{backtraces} of past program points, TSan requires us to \emphA{instrument} \emphA{function entries} and \emphA{exits}
    \item Tail calls must be handled with care
  \end{itemize}
\end{frame}

\begin{frame}{A first challenge: exceptions}
  \begin{itemize}
    \item In C, it is easy to instrument function entries and exits
    \item C++ has to take care of exceptions
    \item OCaml has exceptions too:
      \begin{itemize}
        \item Any function can be exited due to an exception
        \item Unlike in C++, exceptions do not unwind the
          stack\footnote[frame]{Fabrice Buoro,
          \href{https://github.com/fabbing/obts_exn}{``OCaml behind the scenes:
        exceptions''}}
      \end{itemize}
  \end{itemize}

  \begin{itemize}
    \item TSan's linear view of the call stack does not hold
  \end{itemize}
\end{frame}


\newcommand\exnDiagram[1]{%
  \providecommand\step{#1}\begin{tikzpicture}[
node distance=1mm,
scale=1, every node/.append style={transform shape},
base/.style={draw, color=mDarkTeal, semithick},
frame/.style={base, rectangle, rounded corners= 1, minimum width=40mm, minimum height=8mm, inner sep=0, fill=black!3},
scan-frame/.style={frame, draw=mLightBrown, dashed},
stack/.style={rectangle, rounded corners=2},
arrow/.style={base, ->, >=stealth, color=mDarkTeal!70},
short-arrow/.style={arrow, shorten <=2mm, shorten >=2mm},
long-arrow/.style={arrow, shorten <=-1mm, shorten >=-2mm}
]

\pgfdeclarelayer{background};
\pgfsetlayers{background, main};

\node[frame] (frame-f) { f };
\coordinate[below=3 of frame-f] (stack-end);

\ifthenelse{\step=1}{
\node[frame, below=of frame-f] (frame-g) { g };
\node[frame, below=of frame-g] (frame-h) { h };
\node[frame, below=of frame-h] (frame-i) { i };
}{}

\ifthenelse{\step=3}{
\node[scan-frame, below=of frame-f] (frame-g) { g };
\node[scan-frame, below=of frame-g] (frame-h) { h };
\node[scan-frame, below=of frame-h] (frame-i) { i };
}{}

\ifthenelse{\step=2}{
\node[frame, below=of frame-f] (frame-race) { race };
}{}
  
\begin{pgfonlayer}{background}
\node [stack, fit=(frame-f) (stack-end)] {};
\end{pgfonlayer}

\end{tikzpicture}
%
}

\begin{frame}[fragile]{A first challenge: exceptions}
  \begin{columns}[T]
    \begin{column}{0.6\textwidth}
%      \minipage[c][0.66\textheight][s]{\columnwidth}
      \lstsetOCaml
      \begin{lstlisting}
let race () = (* ... *)

let i () = raise Exit%*\begin{tikzpicture}[remember picture,overlay] \coordinate (src-i); \end{tikzpicture}*)
let h () = i ()
let g () = h ()
let f () =
  try g () with | Exit -> race ()%*\begin{tikzpicture}[remember picture,overlay] \coordinate (src-handler); \end{tikzpicture}*)
      \end{lstlisting}
      \onslide*<1,3>{
        \begin{tikzpicture}[remember picture,overlay]
          \coordinate [shift={(5pt,2pt)}] (start) at (src-i);
          \draw [<-,>=stealth,color=mLightBrown,thick] (start) -- ++(6mm,0);
        \end{tikzpicture}
      }
      \onslide*<2>{
        \begin{tikzpicture}[remember picture,overlay]
          \coordinate [shift={(5pt,2pt)}] (start) at (src-handler);
          \draw[<-,>=stealth,color=mLightBrown,thick] (start) -- ++(6mm,0);
        \end{tikzpicture}
      }
      \medskip
      \begin{itemize}
        \item[>>] TSan backtrace:
          \begin{itemize}
              \onslide*<2,4>{
              \item[-] race
              }
              \onslide*<1>{
              \item[-] i
              \item[-] h
              \item[-] g
              }
              \onslide*<2>{
              \item[-] \textcolor{mLightBrown}{i}
              \item[-] \textcolor{mLightBrown}{h}
              \item[-] \textcolor{mLightBrown}{g}
              }
            \item[-] f
          \end{itemize}
      \end{itemize}
%      \end{minipage}
    \end{column}
    \begin{column}{0.4\textwidth}
      \centering
      \only<1>{\exnDiagram{1}}%
      \only<2>{\exnDiagram{2}}%
      \only<3>{\exnDiagram{3}}%
      \only<4>{\exnDiagram{2}}%
    \end{column}
  \end{columns}
\end{frame}


\begin{frame}{A new challenger has arrived: Effect handlers}
  \begin{columns}
    \begin{column}{0.5\textwidth}
      \begin{itemize}
        \item Effect handlers are a \emphA{generalisation of exceptions}: \emphA{perform}-ing an effect jumps to the associated effect handler, and additionally, a delimited continuation makes it possible to \emphA{resume} a computation
          \footnote[frame]{\href{https://arxiv.org/pdf/2104.00250.pdf}{KC Sivaramakrishnan et al, Retrofitting Effect Handlers onto OCaml, PLDI 2021}}
        \item As with exceptions, we must signal to TSan the frames that are \emphA{exited} when an effect is \emphA{performed}, and \emphA{re-entered} when a continuation is \emphA{resumed}
      \end{itemize}
    \end{column}
    \begin{column}{0.5\textwidth}
      \mintocaml{src/effects.ml}
    \end{column}
  \end{columns}
\end{frame}


\begin{frame}{Final boss: The OCaml memory model}
  \begin{itemize}
    \item TSan can detect data races in programs following the C11 memory model
    \item OCaml's memory model\footnote[frame]{\href{https://kcsrk.info/papers/pldi18-memory.pdf}{Dolan et al., \emph{Bounding Data Races In Space and Time}, PLDI 2018}} is different from the C11 one
      \begin{itemize}
        \item It offers more guarantees, such as \emphA{Local Data Race Freedom implies Sequential Consistency} (LDRF-SC)
      \end{itemize}
    \item To enforce the OCaml memory model, some operations are implemented particularly, and special instructions are inserted in the code
      \begin{itemize}
        \item \emphA{Bounding data race in space and time} (LDRF-SC) requires fences at strategic positions
        \item OCaml's runtime, written in C, use strong instructions to prevent \emphB{Undefined Behavior} at C level
      \end{itemize}
    %\item This also prevents TSan to detect some data races (their scope is limited but they are still data races!)
    %\item So we don't always let know of all operations, or all operations details, so TSan can detect race in OCaml Memory Model
  \end{itemize}
\end{frame}

\begin{frame}{Final boss: The OCaml memory model}
  \begin{columns}
    \begin{column}{0.45\textwidth}
      \emphA{OCaml}
      \color{mComplA}{\rule[1em]{\textwidth}{0.1pt}}
    \end{column}
    \begin{column}{0.45\textwidth}
      \emphA{C analogous}
      \color{mComplA}{\rule[1em]{\textwidth}{0.1pt}}
    \end{column}
  \end{columns}
  \medskip
  \begin{columns}
    \begin{column}{0.45\textwidth}
      \mintocaml[lastline=9]{src/simple_race.ml}

      \greencheck{} Well-defined behavior
    \end{column}
    \begin{column}{0.45\textwidth}
      \alt<2->{\mintc{src/simple_race_c11atomics.c}}{\mintc{src/simple_race.c}}

      \alt<2->{\greencheck{} Well-defined behavior}{\redwrong{} Undefined behavior}
    \end{column}
  \end{columns}
\end{frame}


\begin{frame}{Final boss: The OCaml memory model}
  \begin{itemize}
    \item TSan will not detect data races on C11 atomic operations
    \item We do not signal the ``real'' operations to TSan
    \item Instead, we \emphA{map} OCaml memory operations to C11 memory operations so that TSan detects OCaml data races.
  \end{itemize}
\end{frame}
