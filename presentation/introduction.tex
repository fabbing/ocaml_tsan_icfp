\begin{frame}{Goal of this talk}
  \begin{itemize}
    \item What is ThreadSanitizer (TSan) and how is it useful?
    \item What is required to integrate TSan to OCaml programs?
  \end{itemize}
\end{frame}

\begin{frame}{Finally, we can have data races too}
  \begin{columns}
    \begin{column}{0.5\textwidth}
      A \emphA{data race} is a race condition defined by:
      \begin{itemize}
        \item Two accesses are made to the same memory location\,
        \item At least one of them is a write, and
        \item No order is enforced between them.
      \end{itemize}
      \medskip
      \begin{footnotesize}
      Event ordering is formalized in terms of a partial order called \emphB{happens-before}. It is defined by the OCaml 5 memory model.\\
      \end{footnotesize}
      \medskip
      Data races are:
      \begin{itemize}
        \item Hard to detect (possibly silent)
        \item Hard to track down
      \end{itemize}
    \end{column}
    \begin{column}{0.5\textwidth}
      \includegraphics[trim=800 0 150 0, clip, width=1\textwidth]{pictures/camel\_racing.jpg}
    \end{column}
  \end{columns}
\end{frame}

\begin{frame}{Data race example}
  \begin{columns}[T]
    \begin{column}{0.4\textwidth}
      \mintocaml[firstline=1,lastline=15]{src/simple\_race\_on\_the\_slide.ml}
    \end{column}
    \begin{column}{0.4\textwidth}
      \flushright
      \visible<2->{%
        \begin{tikzpicture}[
  base/.style={font=\small\ttfamily},
  arr/.style={base, -{latex}, thick, color=mComplA},
  ev/.style={base, minimum width=4em},
  frame/.style={base, color=gray},
  empirical/.style={color=mLightBrown}
]
  \node[ev] (wa) at (0, 0) { a := 1 };
  \node[ev] (rb) at (0, -1) { !b };
  \node[ev] (wb) at (0, -2.5) { b := 1 };
  \node[ev] (ra) at (0, -3.5) { !a };

  \node[frame, fit=(wa) (rb), draw,dashed, label={[anchor=north west, frame] above right:\texttt{d1}}] {};
  \node[frame, fit=(wb) (ra), draw,dashed, label={[anchor=north west, frame] above right:\texttt{d2}}] {};

  \draw[arr] (wa) to (rb);
  \draw[arr] (wb) to (ra);
  \draw<3->[arr, empirical] (rb) to (wb);
  \draw<4->[arr, empirical] (ra.west) to[bend left=80] (wa.west);
\end{tikzpicture}
%
      }
    \end{column}
  \end{columns}
\end{frame}

\begin{frame}{ThreadSanitizer (TSan)}
  \begin{itemize}
    \item \emphA{Runtime} data race detector (dynamic analysis, not static!)
    \item Initially developed for C++ by Google, now supported in
      \begin{footnotesize}
        \begin{itemize}
          \item C, C++ with GCC and clang
          \item Go
          \item Swift
        \end{itemize}
      \end{footnotesize}
    \item Battle-tested, already found\footnote[frame]{Numbers August 2015}
      \begin{footnotesize}
        \begin{itemize}
          \item 1200+ races in Google’s codebase
          \item ~100 in the Go stdlib
          \item 100+ in Chromium
          \item LLVM, GCC, OpenSSL, WebRTC, Firefox
        \end{itemize}
      \end{footnotesize}
      \bigskip
    \item Requires to compile your program specially
  \end{itemize}
\end{frame}

\begin{frame}{}
  \vspace{6pt}
  \begin{columns}[c]
    \begin{column}{0.4\textwidth}
      \alt<4>{%
        \mintocaml[firstline=1,lastline=15,highlightlines={4, 9}]{src/simple\_race\_on\_the\_slide.ml}
      }{%
        \mintocaml[firstline=1,lastline=15]{src/simple\_race\_on\_the\_slide.ml}
      }%
    \end{column}
    \begin{column}{0.67\textwidth}
      \visible<2-4>{%
        \temporal<3>{%
          \minttxt[firstline=2,breaklines=false,fontsize=\fontsize{5.8pt}{6.2pt}\selectfont,frame=single]{src/tsan\_output.txt}%
        }{%
          \minttxt[highlightlines={2, 17},firstline=2,breaklines=false,fontsize=\fontsize{5.8pt}{6.2pt}\selectfont,frame=single]{src/tsan\_output.txt}%
        }{%
          \minttxt[highlightlines={4, 11},firstline=2,breaklines=false,fontsize=\fontsize{5.8pt}{6.2pt}\selectfont,frame=single]{src/tsan\_output.txt}%
        }%
      }%
    \end{column}
  \end{columns}
\end{frame}

\begin{frame}{}
  \begin{columns}
    \begin{column}{0.3\textwidth}
      \mintocaml[firstline=4,lastline=16]{src/simple\_race\_mutex.ml}
    \end{column}
  \end{columns}
\end{frame}

%% Outline frame
%\begin{frame}{Outline}
%  \tableofcontents
%\end{frame}
