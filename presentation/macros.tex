\setminted{highlightcolor=yellow!33,breaklines,fontsize=\footnotesize,baselinestretch=0.8}
\setminted[lisp]{stripall=true}
\setminted[objdump]{tabsize=2}

% Commands
\newcommand\mintocaml[2][]{%
\expandafter\inputminted\expandafter[#1]{ocaml}{\detokenize{#2}}%
}
\newcommand\mintobjdump[2][]{%
\expandafter\inputminted\expandafter[#1]{objdump}{\detokenize{#2}}%
}
\newcommand\mintcmm[2][]{%
\expandafter\inputminted\expandafter[#1]{lisp}{\detokenize{#2}}%
}
\newcommand\mintlinear[2][]{%
\expandafter\inputminted\expandafter[#1]{text}{\detokenize{#2}}%
}
\newcommand\mintc[2][]{%
\expandafter\inputminted\expandafter[#1]{c}{\detokenize{#2}}%
}
\newcommand\mintcpp[2][]{%
\expandafter\inputminted\expandafter[#1]{cpp}{\detokenize{#2}}%
}
\newcommand\mintasm[2][]{%
\expandafter\inputminted\expandafter[#1]{gas}{\detokenize{#2}}%
}
\newcommand\minttxt[2][]{%
\expandafter\inputminted\expandafter[#1]{text}{\detokenize{#2}}%
}

\newcommand\listlang[4][]{%
% #2 -> language
% #3 -> source file
% #4 -> caption
\begin{listing}
#2[#1]{#3}
\caption{#4}
\end{listing}%
}

\newcommand\listocaml[3][]{%
% #2 -> source file
% #3 -> caption
\listlang[#1]{\mintocaml}{#2}{#3}%
}

\newcommand\listobjdump[3][]{%
% #2 -> source file
% #3 -> caption
\listlang[#1]{\mintobjdump}{#2}{#3}%
}

\newcommand\listcmm[3][]{%
% #2 -> source file
% #3 -> caption
\listlang[#1]{\mintcmm}{#2}{#3}%
}

\newcommand\listlinear[3][]{%
% #2 -> source file
% #3 -> caption
\listlang[#1]{\mintlinear}{#2}{#3}%
}

\newcommand\listc[3][]{%
% #2 -> source file
% #3 -> caption
\listlang[#1]{\mintc}{#2}{#3}%
}

\newcommand\listasm[3][]{%
% #2 -> source file
% #3 -> caption
\listlang[#1]{\mintasm}{#2}{#3}%
}


%\addtobeamertemplate{block alerted begin}{%
%    \setlength{\textwidth}{0.8\textwidth}
%}{}
%
%\addtobeamertemplate{block example begin}{%
%    \setlength{\textwidth}{0.8\textwidth}
%}{}

\colorlet{FancyVerbHighlightColor}{blue}

\newcommand{\redwrong}{{\color{red}\faTimes}}
\newcommand{\greencheck}{{\color{green!50!black}\faCheck}}
\newcommand{\ballotBox}{\faSquareO}
\newcommand{\checkedBallotBox}{\faCheckSquareO}
\newcommand{\consequence}{\color{red}{\rotatebox[origin=c]{180}{\faReply}}}


\lstset{
  basicstyle=\ttfamily\scriptsize,
  columns=fullflexible,
  captionpos=b,                    % sets the caption-position to bottom
  escapeinside={\%*}{*)},          % if you want to add LaTeX within your code
%  frame=single,	                   % adds a frame around the code
  keepspaces=true,                 % keeps spaces in text, useful for keeping indentation of code (possibly needs columns=flexible)
% language=C,                 % the language of the code
%  numbers=left,                    % where to put the line-numbers; possible values are (none, left, right)
%  numbersep=5pt,                   % how far the line-numbers are from the code
%  numberstyle=\scriptsize\color{darkgray}, % the style that is used for the line-numbers
%  rulecolor=\color{black},         % if not set, the frame-color may be changed on line-breaks within not-black text (e.g. comments (green here))
  mathescape=true,
  literate={->}{$\to$}{1},
}

\definecolor{blueish}{HTML}{214c95}
\definecolor{orangeish}{HTML}{d6772b}

\newcommand\lstsetOCaml{
  \lstset{
    language=[Objective]Caml,
    keywordstyle=\color{blueish},
    commentstyle=\color{orangeish},
    }
}

\newcommand\lstsetNone{
  \lstset{language=,}
}

\definecolor{mComplA}{HTML}{384E94}
\definecolor{mComplB}{HTML}{944B38}

\newcommand\emphA[1]{%
  \textcolor{mComplA}{\textbf{#1}}%
}

\newcommand\emphB[1]{%
  \textcolor{mComplB}{\textbf{#1}}%
}

\newcommand\hruleA{%
  \color{mComplA}{\rule[1em]{\textwidth}{0.1pt}}%
}
